\section{Introduction}


Indoor builders are always interested in finding ways to improve their model
flight times. I started the \MM\ project
(https://rblack42.github.io/math-magik) {cite}`mmagic` in 2020 as part
of my plan to build a Python application to assist model builders in designing
and flying indoor model airplanes. The initial effort, published in the 2021
issue of the NFFS Symposium {cite}`rblack` focused on using \osc\ and some
supporting Python code to design a model and analyze the proposed design to
calculate weight and balance data. This article extends that project by
providing a Python application that can help predict flight times of a model.

This article revisits work presented by Doug McLean in the 1976 issue of the
NFFS Symposium. McLean is a highly respected aeronautical engineer, and a
retired Technical Fellow at Boeing where he worked for many years. He has
authored a nice book titled {\it Understanding Aerodynamics} ({cite}`mclean
`3`) which I am using as a reference for this project. My goal here is to
provide better documentation on the theory behind Doug's scheme and Python code
that implements his method. All of the theory and code is available in a more
complete form on the companion website at {\bf
https://rblack42.github.io/math-magik-flight-time} \cite{mmtime}

Due to space limitations in the Symposium format, the code you see here is
abbreviated. If you are interested in trying out the program I developed, I
suggest that you download the complete code from the project website at {\bf
https://rblack42.github.io/math-magic-flight-time/release.zip}.


