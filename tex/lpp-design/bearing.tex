\subsection{Bearing Geometry}

We will use a simple aluminium sheet bearing for this model. The basic geometry
is shown in \ref{fig:bearing-geometry}.

\importtikzfigure{bearing-geometry}{Bearing Geometry}

We will use this diagram to create a 2D shape, then extrude it into a 3D shape.
The prop shaft will sit at the bottom of this shape, so we will add cylindrical
parts at the two bottom edges to enclose the shaft. All of this will be wrapped
up into a general purpose module we can reuse in other projects.

The important parameters in the digram are the length, height, material thickness, and
the slope of the front part of the bearing. All bends are made at a radius of
twice the material thickness. We can choose the length of the top part, or the
length of the front flat and everything else can be calculated with a bit of
trigonometry.

From this diagram, we can derive these equations:

\begin{eqnarray}
  d &=& h - 2t, \\
  b sin(\alpha) + c &=& l - 4t, \\
  b cos(\alpha) + a &=& d
\end{eqnarray}

That give three equations in three unknowns. A bit of {\it SymPy} will give us
the formulas we need to generate this shape.

\begin{sympysilent}
a,b,c,d,h,l,t,alpha = sympy.symbols("a b c d b l t alpha")

eq1 = d - h + 2*t
eq2 = b*sympy.sin(alpha) + c - l + 4*t
eq3 = b * sympy.cos(alpha) + a -d;

sol = sympy.solve([eq1,eq2,eq3], [a,b,c])
\end{sympysilent}

\begin{equation}
\left\{ a : - d \cos{\left(\alpha \right)} + d - 2 t \cos{\left(\alpha \right)}, \  b : d + 2 t, \  c : - d \sin{\left(\alpha \right)} + l - 2 t \sin{\left(\alpha \right)} - 4 t\right\}
\end{equation}


