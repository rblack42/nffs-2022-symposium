\subsection{Final Assembly}

Now that we have all of the major components defined, it is time to put things
together and see the completed model.

\subsubsection{Mounting Components}

The wing will be mounted on top of two hard balsa posts using paper tubes.
These wing posts are glued to the motor stock on the bottom and rounded at the
top to slide into paper tubes that will be glued to the wing structure.

A similar arrangement is used to attach the stabilizer on top of the tail boom.

The rudder is glued to the bottom on the tail boom, but is offset to provide for
a turn during flight. A small stick of balsa will be glued to the rear of the
tail boom and the fin to provide needed support. Should this need adjusting,
the rear attachment can be unglued and repositioned.

\subsubsection{Paper Tubes}

Paper tubes are formed on a mandrel using a strip of tissue soaked in thin
glue and twisted around that mandrel. The resulting tube, after pulling off of
the mandrel, will be stiff enough to provide the support needed. The posts must
be carefully sanded to ensure a tight fit.

The code that creates a paper tube, it is just a very skinny hollow cylinder.

Figure \ref{fig:wing-mount-tube.png} shows an image showing how the tubes are attached
to both the wing and stabilizer:

\importimage{wing-mount-tube.png}{Wing Mount Tube}

\subsubsection{Mounting Posts}

The posts used to attach the wing and stabilizer are simple sticks of hard balsa with
rounded tops that are sized for a tight fit in the paper tubes.

Since we need several mounting posts, the code that generates the post is
placed in another module:

To attach the wing, we need to attach two posts to the motor stick. Figure
\ref{fig:round_post.png} shows what they look like,


\importimage{round_post.png}{Mounting Post}

We use a very short post for the fin as well.

Both the stabilizer and fin mount on the tail boom.

\subsubsection{Stabilizer}

The stabilizer is mounted on top of the tail boom, using small posts and paper
tubes to allow for minor adjustments when flying.

\subsubsection{Vertical Fin}

The rudder is simple attached to the bottom of the tail boom, Since that side
of the boom is canted by the trimming we performed earlier, we need to rotate
the rudder slightly during positioning.  However, we allow for  left rudder
attachment to provide a left turn flight path.

\subsubsection{Adding the Propeller}

We still need to set up the propeller. The wire parts need to be positioned,
then the propeller translated into place. I rotated the prop a bit so the final
image looks nicer. What you get us shown in the figure at the beginning of this
article.


